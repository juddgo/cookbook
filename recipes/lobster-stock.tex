\begin{recipe}{Lobster Stock}{3 qts}{2 hours}
\ing[4]{}{lobster carcasses}
Remove gills (and sand sac?) as well as any singed shell.
\ing[2]{}{onions, chopped}
\ing[6]{}{carrots, chopped}
\ing[4]{tbsp}{olive oil}
Sautee in a large stock pot until tender.
\ing[4]{cloves}{garlic, chopped}
\ing[6]{}{plum tomatoes, canned}
Add garlic and shells to stock pot and cook for a few minutes
\ing[1]{c.}{white wine}
Add white wine and cook off the alcohol.
\ing[4]{}{bay leaves}
\ing[]{}{black pepper, ground}
Add herbs and spices. Here we use bay leaves and black pepper, but
parsley is called for.
\ing[]{}{water} Add enough water to cover the shells by a few inches. Bring to a boil, and simmer for 60-90 minutes.\\
\freeform I also added some of the cooking water from the lobster boil.  Remove shells and pour stock through a medium filter or cheese cloth.  Return stock to the stove.\\
\ing[]{}{thyme}
\ing[]{}{red pepper flakes}
\ing[]{}{oregano}
Simmer the stock and season to taste.\\
\freeform The stock can be frozen in an ice cube tray and stored.
\end{recipe}

%%\rule{\textwidth}{0.05pt}
%\newstep Wait fo two weeks.  Skim periodically.
%\ing[\fr12]{c.}{salt}
%% \freeform
