%================================================================
%================================================================
%================================================================
%================================================================
%\chapter{Sam's Recipes}
%% \begin{recipe}{Basic Beef Chili}{about 3 quarts}{3 hours}
%% \freeform This recipe was taken from ``The Joy of Cooking.''
%% \ing[\fr12-1]{c.}{chili powder}
%% Toast the chili powder in a dry skillet over medium-high heat.
%% \ing[1]{tbsp}{olive oil}
%% \ing[3]{lbs}{ground or cubed beef}
%% Add oil to a skillet and brown the beef.  Transfer to a covered pot. 
%% \ing[1]{tbsp}{olive oil}
%% \ing[2]{}{large onions, minced}
%% \ing[10]{}{cloves garlic, minced}
%% \ing[7]{}{fresh jalape\~{n}os, stemmed, seeded and minced}
%% Add more oil to the skillet and soften the onions, garlic and jalape\~{n}os for 6-8 minutes.  Add to the beef.
%% \newstep
%% Add the chili powder to the meat mixture and stir over medium-high heat for two minutes.
%% \ing[1]{}{28 oz. can plum tomatoes, with juice}
%% \ing[1]{tbsp}{red wine vinegar}
%% \ing[6]{c.}{water}
%% Add the tomatoes, vinegar and water and simmer, uncovered, for as long as possible.  Add salt to taste.
%% \end{recipe}
%================================================================
%================================================================
%================================================================
%================================================================
%================================================================
%================================================================
\newpage
\begin{recipe}{Basic Rib Rub}{about 1\fr12 cups}{15 minutes}
\freeform Taken from \emph{\path{allrecipes.com}}. We used this recipe with Sam's Chili powder \#4, recipe number ~\ref{sam4} to make pork spare ribs in the smoker.  May 1, 2011.
\ing[\fr12]{cup}{packed brown sugar}
\ing[2]{tbsp}{chili powder}
\ing[1]{tbsp}{paprika}
\ing[1]{tbsp}{freshly ground black pepper}
\ing[2]{tbsp}{garlic powder}
\ing[2]{tsp}{onion powder}
\ing[2]{tsp}{kosher salt}
\ing[2]{tsp}{ground cumin}
\ing[1]{tsp}{ground cinnamon}
\ing[1]{tsp}{jalape\~{n}o seasoning salt (optional)}
Combine all ings and store in an air-tight container.
%\freeform\hrulefill
\end{recipe}
%================================================================
%================================================================
%================================================================
\newpage
\begin{recipe}{Basic Barbecue Mop}{about 2 cups}{15 minutes}
\freeform Taken from \emph{\path{allrecipes.com}}.
\ing[1]{cup}{apple cider}
\ing[\fr34]{cup}{apple cider vinegar}
\ing[1]{tbsp}{onion powder}
\ing[1]{tbsp}{garlic powder}
\ing[2]{tbsp}{lemon juice}
\ing[1]{}{jalape\~{n}o pepper, finely chopped (optional)}
\ing[3]{tbsp}{hot pepper sauce}
\ing{}{kosher salt}
\ing{}{black pepper}
Combine all ings and mix.
\freeform We didn't use hot sauce, but we did use the jalape\~{n}o.  We used this mop on pork spare ribs rubbed with the basic rib rub \ref{}.  It was fine -- nothing to get too excited about.  Similar to what we normally used, and is thus the basic rib rub.
\end{recipe}
%================================================================
%================================================================
%================================================================
%% \newpage
%% \begin{recipe}{Sri Lankan Curry Powder}{about a \fr12 cup}{20 minutes}
%% \freeform See \small\emph{\path{www.srilankacooking.com/2009/02/how-to-make-sri-lankan-curry-powder.html}}.
%% \ing[6]{tbsp}{coriander seeds}
%% \ing[2]{tbsp}{cumin seeds}
%% \ing[1]{tsp}{fennel seeds}
%% \ing[1]{tsp}{mustard seeds}
%% \ing[1]{}{\unit[7]{cm} cinnamon stick (crumbled)}
%% \ing[4]{}{whole cloves}
%% \ing[1]{tsp}{ground cardamom}
%% \ing[1]{tsp}{curry leaf powder}
%% \ing[1]{tsp}{whole black peppercorns}
%% Combine ings and toast in a dry skillet on medium heat for 5-6 minutes, or ``until aromatic and golden brown.''  Let cool and grind to a fine powder. 
%% \freeform The recipe called for 4 cardamom pods, but we only had powder.  The recipe also called for 5 dried curry leaves, but we used 1 tsp of a ground curry leaf powder, which contained many ings other than curry leaves.  We found that the cinnamon and cloves didn't grind completely.\\

%% We would like more cinnamon and clove, and Julie thought it was heavy on the cumin.  Sam would have liked something spicy in it, perhaps some chilis or red pepper, and would have preferred straight curry leaves.  We are dehydrating fresh curry leaves for the next batch.
%% \end{recipe}
%% %================================================================
%% %================================================================
%% %================================================================
%% \newpage
%% \begin{recipe}{Sam's Curry Powder \#2}{about a \fr12 cup}{20 minutes}
%% \freeform This one was only loosely based on an online recipe, and not similar enough to bother with the citation.
%% \ing[3]{tbsp}{cumin seeds}
%% \ing[1]{tbsp}{fennel seeds}
%% \ing[1]{}{\unit[5 \nicefrac{1}{2}]{in.}\ cinnamon stick (crumbled)}
%% \ing[1]{tbsp}{whole cloves}
%% \ing[\fr23]{tbsp}{cardamom powder}
%% \ing[1]{tbsp}{fenugreek seeds}
%% \ing[\fr23]{tbsp}{coriander seeds}
%% \ing[1]{tbsp}{chili powder (homemade)}
%% \ing[6]{}{dried curry leaves}
%% \ing[\fr23]{tbsp}{whole black peppercorn}
%% \ing[\fr13]{tbsp}{mustard seeds}
%% Combine ings and toast in a dry skillet, moving constantly, on medium heat for about 5 minutes, or ``until aromatic and golden brown.''  Let cool and grind to a fine powder. 
%% \freeform 
%% \end{recipe}
%% %================================================================
%% %================================================================
%% %================================================================
%% \newpage
%% \begin{recipe}{Sam's Curry Powder \#3}{about a \fr12 cup}{20 minutes}
%% \ing[3]{tbsp}{cumin seeds}
%% \ing[2]{tbsp}{chili powder (homemade)}
%% \ing[1]{tbsp}{whole cloves}
%% \ing[1]{tbsp}{fenugreek seeds}
%% \ing[1]{tbsp}{coriander seeds}
%% \ing[1]{tbsp}{whole black peppercorn}
%% \ing[\fr23]{tbsp}{mustard seeds}
%% \ing[\fr13]{tbsp}{fennel seeds}
%% \ing[\fr13]{tbsp}{cardamom powder}
%% \ing[8]{}{dried curry leaves}
%% \ing{}{\unit[7 \nicefrac{1}{2}]{in.}\ cinnamon stick (crumbled)}
%% Combine ings and toast in a dry skillet, moving constantly, on medium heat for about 5 minutes, or ``until aromatic and golden brown.''  Let cool and grind to a fine powder. 
%% \freeform Not very good.  Too much cinnamon and cloves.  Not enough spice.  For some reason it just kind of tasted like peanut butter.
%% \end{recipe}
%================================================================
%================================================================
%================================================================
\newpage
\begin{recipe}{Chinese Slow Cooked Pork Shoulder}{3 lbs.}{6 hours}
\freeform Adapted from \emph{\path{foodnetwork.com}}.
\ing[3]{lbs}{pork shoulder, trimmed}
\ing[1]{tsp}{Chinese five-spice powder}
\ing[1]{tsp}{kosher salt}
Rub the pork with the five-spice powder and salt.
\ing[3]{cups}{chicken broth}
\ing[1]{cup}{dark soy sauce}
\ing[\fr14]{cup}{dark brown sugar, packed}
\ing[2]{tbsp}{sesame oil}
\ing[\fr12]{tsp}{red pepper, crushed}
Add the chicken broth, soy sauce, brown sugar, sesame oil and red pepper flakes to the slow cooker.  Stir to dissolve.
\ing[4]{}{scallions, cut into 2 in. pieces}
\ing[1]{}{garlic head, halved}
\ing[1]{}{2 in. knob unpeeled fresh ginger, thinly sliced}
\ing[8]{}{shiitake mushrooms, dried (optional)}
Add the scallions, garlic, ginger, mushrooms and pork shoulder to the slow cooker.  Turn a few times to coat.  Cover the cooker, and cook on high for 4 hours.  Turn down to low and cook for another two hours. 
\ing{}{Chinese noodles (Ramen works well)}
Remove pork and let rest for 15 minutes.  Skim the cooking liquid and serve over the pork and noodles.
\freeform We made this recipe without shiitake mushrooms. In addition, we lacked fresh ginger so used crushed ginger from a jar. After several hours, the pork was boiling a good deal so we turned it to low earlier than prescribed in the recipe. We strained the sauce to remove the garlic skin and other solids, and skimmed off most of the fat. We served the pork pulled with sauce on top of ramen noodles, which was excellent. Some fresh chives or chopped scallions on top could be a nice addition.
\end{recipe}
%================================================================
%================================================================
%================================================================
%% \newpage
%% \begin{recipe}{Roasted Kale with Chili Powder}{\unit[4]{cups}}{\unit[15]{min.}}
%% \freeform From \emph{\path{allrecipes.com}}.
%% \ing[4]{cups}{kale}
%% \ing[1]{tbsp}{extra virgin olive oil}
%% Wash and stem the kale. Toss in a large bowl with the olive oil.
%% \ing[1]{tbsp}{chili powder}
%% \ing[\fr12]{tsp}{kosher salt}
%% Add the chili powder and salt and mix until evenly coated.  Roast on a baking sheet in a 400\0 oven for about 5 minute.  Stir the kale and roast for another few minutes.
%% \end{recipe}
%================================================================
%================================================================
%================================================================
\newpage
\begin{recipe}{Clams Steamed in White Wine}{\unit[1]{dozen}}{\unit[3]{hrs.}}
\freeform From \emph{\path{http://whatscookingamerica.net/ClamsSteamer.htm}}.
\ing[1]{dozen}{littleneck clams}
\ing[4]{cups}{water}
\ing[3]{tsp}{Kosher or sea salt}
\ing[\fr14]{cup}{corn starch}
Soak clams in water, salt and corn starch mixture for a few hours.  Scrub and rinse.
\ing[1]{tbsp}{butter}
\ing[\fr12]{}{onion, chopped}
\ing[2]{}{cloves garlic, chopped}
Melt the butter in a pot.  Add onion and garlic, and soften.
\ing[1]{cup}{white wine}
\ing[\fr14]{tsp}{red pepper flakes}
Add the wine and red pepper flakes and bring to a boil.  Add the clams, reduce heat, and cover.  Cook for 5--10 minutes or until the clams open.  Strain the cooking liquid and serve with the clams, along with thick ``nice'' bread for dipping, e.g., a baguette.
\end{recipe}
%================================================================
%================================================================
%================================================================
%================================================================
%================================================================
%================================================================
%% \newpage
%% \begin{recipe}{Pan-fried Almond-encrusted catfish}{4 servings}{\unit[20]{min.}}
%% \freeform Great served with Ginger soy sauce.
%% \ing[4]{fillets}{catfish}
%% \ing{}{milk}
%% Cover catfish fillets in milk and let stand while preparing crust.
%% \ing[1]{cup}{flour}
%% \ing[\fr23]{cup}{ground almond slivers}
%% \ing[1]{tsp}{salt}
%% \ing[\fr12]{tsp}{ground black pepper}
%% \ing{pinch}{cayenne pepper}
%% Mix together in a shallow bowl. Remove each catfish fillet from the milk and press into almond crust mixture.
%% \ing[1]{tbsp}{olive oil}
%% \ing[1]{tbsp}{butter}
%% Heat butter and olive oil over medium-high heat (5 or 6 on our stove). Pan fry encrusted catfish fillets for approximately 4 or 5 minutes per side until dark golden brown. Serve with Ginger Soy Sauce.
%% \end{recipe}
%================================================================
%================================================================
%================================================================
\newpage 
\begin{recipe}{Ginger Soy Sauce}{4 servings}{\unit[5]{min.}}
\freeform Great served on meat, fish or vegetables.
\ing[\fr12]{cup}{soy sauce}
\ing[\fr12]{cup}{balsamic vinegar}
\ing[1]{tsp}{crushed ginger or fresh grated ginger}
\ing[2]{tbsp}{water}
\ing[3--4]{tbsp}{chopped scallions}
\ing[\fr34]{tsp}{sugar}
\freeform Whisk all ings together until sugar is dissolved.
\end{recipe}
%================================================================
%================================================================
%================================================================
\newpage
\begin{recipe}{Chilled Beet Soup with Kefir and Chives}{8 servings}{\unit[1]{hr.}}
\freeform Adapted from crumpetsandcakes.blogspot.com.
\ing[4--6]{}{beets, peeled and chopped}
\ing[]{}{chicken stock}
Just cover beets with chicken stock and boil for about 20 minutes until tender. Season with salt. Set aside to cool.
\ing[3]{}{large radishes, finely chopped}
\ing[\fr12]{}{large English cucumber, about one cup finely chopped}
\ing[4]{tbsp}{chopped chives}
\ing[4]{tbsp}{chopped fresh dill}
Reserve half of chopped vegetables and herbs. Combine the other half with cooled beet and stock mixture in a blender and pur\'{e}e. Mix in reserved chopped vegetables and herbs. Refrigerate until chilled.
\ing[12]{oz}{plain kefir}
\ing[1]{tbsp}{sugar (optional)}
\ing{}{salt and pepper to taste}
When beet mixture is chilled, add sugar, salt and pepper to taste. Mix in desired amount of kefir to each serving and garnish with finely chopped cucumber and chives.
\end{recipe}
%================================================================
%================================================================
%================================================================
%================================================================
%\newstep
%\freeform\rule{\textwidth}{0.05pt}\newpage
%\freeform\rule{\textwidth}{0.05pt}
